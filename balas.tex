\documentclass{article}
\usepackage[utf8]{inputenc}
\usepackage[portuguese]{babel}
\usepackage{graphicx} % Required for inserting images
\usepackage{url}
\usepackage{csquotes}
\usepackage{amsmath}
\usepackage{amsfonts}
\usepackage[
maxbibnames=99,
backend=bibtex,
%style=alphabetic,
sorting=ynt
]{biblatex} %Imports biblatex package
\addbibresource{bibliografia.bib}

\title{Algoritmo de Balas aplicado ao TSP com janelas de tempo e carro elétrico}

\begin{document}

\maketitle

%\section*{Identificação}

\noindent{\bf Orientador}:  Álvaro Junio Pereira Franco - \url{alvaro.junio@ufsc.br}\\

\noindent{\bf Estudante}: Felipe Lourenço da Silva - \url{felipe.lourenco@grad.ufsc.br}\\

\noindent{\bf Departamento}: Informática e Estatística\\

\noindent{\bf Centro de ensino:} Tecnológico\\

\noindent{\bf Título do projeto:} Modelos e algoritmos para variações modernas do problema de roteamento de veículos\\

\noindent{\bf Fonte financiadora:} Conselho Nacional de Desenvolvimento Científico e Tecnológico - Termo de Outorga - Processo: 405247/2023-0 (SIGPEX N. 202104940)

\section*{Resumo}
Adaptamos o algoritmo de Balas que resolve o TSP com janelas de tempo para resolver o TSP com janelas de tempo e carro elétrico.

\section*{O algoritmo de Balas}
O algoritmo de Balas foi apresentado no artigo \cite{Ba}. Uma implementação deste algoritmo foi apresentada em \cite{BS}. Vamos usar a mesma notação que aparece nos artigos citados. O grafo de entrada possui $n$ vértices e é completo. As $n$ cidades são indexadas por $1, 2, \dots, n$. Muitas vezes escrevemos $i$ como o índice da cidade $i$. Uma permutação $\pi$ de $1, 2, \dots, n$ é uma solução do TSP clássico. A posição da cidade $i$ em uma solução viável do problema é denotado por $\pi(i)$. O grafo possui custos nas arestas $t_{ij}$ que representam tanto a distância quanto o tempo entre os vértices $i$ e $j$. O algoritmo de Balas resolve o TSP com a seguinte restrição: dado um inteiro positivo $k$ e qualquer par de cidades $i$ e $j$, se $j \geq i + k$ então em qualquer solução viável do problema, a cidade $i$ deve preceder a cidade $j$, ou seja, $\pi(i) < \pi(j)$. A este problema, vamos nos referir como TSP restrito. O algoritmo consiste de uma redução do TSP restrito para o problema de encontrar um caminho de menor custo em um grafo dirigido acíclico.

Existem propriedades importantes que valem destacar. A primeira delas determina as cidades candidatas a ocupar uma determinada posição em uma solução viável, $\pi^{-1}$. As cidades candidatas a ocupar a posição $i$ de uma solução viável são aquelas dentro do intervalo de inteiros $\{\max\{1, i - k + 1\}, \dots, \min\{i + k - 1, n\}\}$. Podemos então escrever que uma cidade candidata $\pi^{-1}(i)$ atende a seguinte restrição $$\max\{1, i-k+1\} \leq \pi^{-1}(i) \leq \min\{i+k-1,n\}.$$ Para qualquer cidade $j$, a posição de $j$ em uma solução viável também está definida dentro do seguinte intervalo de inteiros $\{\max\{1, j-k+1\}, \dots \min\{j + k -1,n\}\}$, ou seja, $$\max\{1, j-k+1\} \leq \pi(j) \leq \min\{j+k-1,n\}.$$ Portanto, dada a ($i-1$)-ésima visita do caixeiro viajante a algum vértice do grafo, o próximo vértice $j$ a ser visitado poderá ser obtido através da construção de um grafo auxiliar considerando todas as possibilidades de vértices candidatos para a posição $i$ (ou seja, $\pi^{-1}(i) = \max\{1, i-k+1\}, \dots, \min\{i+k-1,n\}$) e considerando todas as posições que uma cidade $j$ pode ocupar (ou seja, $\pi(j) = \max\{1, j-k+1\}, \dots, \min\{j+k-1,n\}$). Caso alguma cidade candidata $j$ para a posição $i$ não esteja dentro do intervalo descrito acima, então essa cidade não deve ser considerada. 

O grafo auxiliar $G^*$ é construído com $n+1$ camadas. O conjunto de vértices da camada $i$ do grafo auxiliar é denotado por $V^*_i$. $V^*_1 = \{s\}$ (vértice fonte $s$). $V^*_{n+1} = \{t=s\}$ (vértice sorvedouro $t$ que é igual a fonte $s$). O número de vértices de uma cada $V^i_i$ é no máximo $(k+1)2^{k-2}$ para $i=2, \dots, n$. O grau de entrada de cada vértice do grafo auxiliar é no máximo $k$.

Em geral, a segunte recorrência calcula o custo de um segmento ótimo começando na cidade 1, passando pelas cidades de um subconjunto $W\subset N$ nas posições $2, \dots, i-1$ e visitando a cidade $j$ na posição $i$: $$C(W,i,j) = \min_{l\in W}\{W \backslash \{l\}, i-1, l\}.$$

No TSP clássico, existem muitas possibilidades para um conjunto $W$ com $i-2$ elementos, a saber, $\binom {n}{i-2}$. No entanto, no TSP restrito, as possibilidades para $W$ são poucas e descritas por pares de subconjuntos cujo tamanho de cada um é limitado por $k$. Os pares são:
$$ S^-(\pi, i):= \{l \in (1, \dots, n): l \geq i, \pi(l) \leq i-1\}$$
e
$$ S^+(\pi, i):= \{h \in (1, \dots, n): h \leq i-1, \pi(h) \geq i\}.$$

Descrevendo os conjuntos acima de outra forma, $S^-(\pi, i)$ é o conjunto de cidades de uma solução viável $\pi$ que possuem índices maiores ou iguais a $i$ e que foram visitadas em alguma posição $1, \dots, i-1$, enquanto que $S^+(\pi,i)$ é o conjunto de cidades de uma solução viável $\pi$ que possuem índices menores ou iguais a $i-1$ e serão visitadas em alguma posição $i, \dots, n$. É demonstrado que $|S^-(\pi, i)| = |S^+(\pi, i)| \leq \lfloor k/2 \rfloor$.

Um caminho de custo mínimo no grafo auxiliar passará por um vértice em cada camada do grafo. Um vértice deste caminho na camada $i$ contém a cidade que é visitada na posição $i$ da solução ótima. Um vértice na camada $i$ do grafo auxiliar é denotado por $(i, j, S^-_{ij}, S^+_{ij})$ sendo $j$ uma cidade candidata à posição $i$, e $S^-_{ij}$ e $S^+_{ij}$ sendo, respectivamente, $S^-(\pi, i)$ e $S^+(\pi, i)$. A construção dos vértices  $(i, j, S^-_{ij}, S^+_{ij})$ da camada $V^*_i$ é realizada considerando todos os candidatos $j$ para a posição $i$. Dessa forma, é possível obter todos os vértices da camada $V^*_i$. Os arcos de $G^*$ conectam vértices de camadas consecutivas. Dois vértices, $(i-1, l, S^-_{i-1, l}, S^+_{i-1, l})$ e $(i, j, S^-_{ij}, S^+_{ij})$ são adjacentes (\emph{compatíveis}) se cada caminho $T_{i,j}$ em $G$ que vai da cidade 1 até a cidade $j$ na posição $i$ pode ser obtido por um caminho $T_{i-1,l}$ em $G$ que vai da cidade 1 até a cidade $l$ na posição $i-1$. Ou seja, basta adicionar $j$ ao caminho $T_{i-1,l}$ para obter $T_{i,j}$. Os vértices de um caminho $T_{i,j}$ estão no conjunto $N(i,j,S^-_{ij}, S^+_{ij}) = (N_{i-1} \setminus S^+_{ij}) \cup S^-_{i,j}$. $N_i$ é uma notação para o conjunto das cidades $\{1, 2, \dots, i\}$. Com isso, vértices $(i-1, l, S^-, S^+)$ e $(i, j, S^-, S^+)$ são compatíveis se e somente se $$N(i, j, S^-_{ij}, S^+_{ij}) = N(i-1, l, S^-_{i-1,l}, S^+_{i-1,l}) \cup \{l\}.$$ Para obter os arcos de maneira eficiente, a última equação é expressa em termos dos conjuntos $S^-_{ij}$, $S^+_{ij}$, $S^-_{i-1, l}$ e $S^+_{i-1, l}$.

\paragraph{Proposição de Balas.} Os vértices $(i-1, l, S^-_{i-1, l}, S^+_{i-1, l})$ e $(i, j, S^-_{ij}, S^+_{ij})$ são compatíveis se e somente se $j \neq l$ e vale uma das seguintes condições.
\begin{itemize}
\item Se $l < i-1$ e $i-1 \in S^-_{i-1, l}$, \\então $S^-_{ij} = S^-_{i-1,l}\setminus\{i-1\}$ e $S^+_{ij} = S^+_{i-1,l} \setminus \{l\}$.
\item Se $l < i-1$ e $i-1 \notin S^-_{i-1, l}$, \\então $S^-_{ij} = S^-_{i-1,l}$ e $S^+_{ij} = S^+_{i-1,l} \setminus \{l\} \cup \{i-1\}$.
\item Se $l = i-1$, \\então $S^-_{ij} = S^-_{i-1,l}$ e $S^+_{ij} = S^+_{i-1,l}$.
\item Se $l > i-1$ e $i-1 \in S^-_{i-1, l}$, \\então $S^-_{ij} = S^-_{i-1,l}\setminus\{i-1\} \cup \{l\}$ e $S^+_{ij} = S^+_{i-1,l}$.
\item Se $l > i-1$ e $i-1 \notin S^-_{i-1, l}$, \\então $S^-_{ij} = S^-_{i-1,l} \cup \{l\}$ e $S^+_{ij} = S^+_{i-1,l} \cup \{i-1\}$.
  
\end{itemize}




O tempo de execução para resolver o problema do caminho mínimo no grafo auxiliar construído previamente é $O(k^22^{k-2}n)$.

\medskip
\printbibliography

\end{document}
